%------------------------------------------------------------------------------
% CV in Latex
% Author : Charles Rambo
% Based off of: https://github.com/sb2nov/resume and Jake's Resume on Overleaf
% Most recently updated version may be found at https://github.com/fizixmastr 
% License : MIT
%------------------------------------------------------------------------------

%\documentclass[A4,11pt]{article}
\documentclass[letterpaper,11pt]{article} %For use in US
\usepackage{latexsym}
\usepackage[empty]{fullpage}
\usepackage{titlesec}
\usepackage{marvosym}
\usepackage[usenames,dvipsnames]{color}
\usepackage{verbatim}
\usepackage{enumitem}
\usepackage[hidelinks]{hyperref}
\usepackage[english]{babel}
\usepackage{tabularx}
\usepackage{tikz}
\input{glyphtounicode}

\begin{comment}
I am by no means a professional when it comes to the CV's/resumes, I have received various trainings on how to write a CV and resume from my high 
school, as well as the Austin College and University of Eastern Finland's
career counseling departments. As I intend to share my CV as a template, I 
feel that it is my responsibility to provide explanations of my work.
\end{comment}


%-----FONT OPTIONS-------------------------------------------------------------
\begin{comment}
The font of the document will impact not just how readable it is, but how it is
perceived. In the "The Craft of Scientific Writing" by Michael Alley, shares a
common fonts for publication as well as their use. I have chosen to use
Palatino for its legibility, some others are given below. There is far too much
about typography to discus here. Note: serif fonts have short projecting
strokes, sans-serif fonts are sans (without) these strokes.
\end{comment}


% serif
 \usepackage{palatino}
% \usepackage{times} %This is the default as well
% \usepackage{charter}

% sans-serif
% \usepackage{helvet}
% \usepackage[sfdefault]{noto-sans}
% \usepackage[default]{sourcesanspro}

%-----PAGE SETUP---------------------------------------------------------------

% Adjust margins
\addtolength{\oddsidemargin}{-1cm}
\addtolength{\evensidemargin}{-1cm}
\addtolength{\textwidth}{2cm}
\addtolength{\topmargin}{-1cm}
\addtolength{\textheight}{2cm}

% Margins for US Letter size
%\addtolength{\oddsidemargin}{-0.5in}
%\addtolength{\evensidemargin}{-0.5in}
%\addtolength{\textwidth}{1in}
%\addtolength{\topmargin}{-.5in}
%\addtolength{\textheight}{1.0in}

\urlstyle{same}

\raggedbottom
\raggedright
\setlength{\tabcolsep}{0cm}

% Sections formatting
\titleformat{\section}{
  \vspace{-4pt}\scshape\raggedright\large
}{}{0em}{}[\color{black}\titlerule \vspace{-5pt}]

% Ensure that .pdf is machine readable/ATS parsable
\pdfgentounicode=1

%-----CUSTOM COMMANDS FOR FORMATTING SECTIONS----------------------------------
\newcommand{\CVItem}[1]{
  \item\small{
    {#1 \vspace{-2pt}}
  }
}

\newcommand{\CVSubheading}[4]{
  \vspace{-2pt}\item
    \begin{tabular*}{0.97\textwidth}[t]{l@{\extracolsep{\fill}}r}
      \textbf{#1} & #2 \\
      \small#3 & \small #4 \\
    \end{tabular*}\vspace{-7pt}
}

\newcommand{\CVPublication}[4]{
  \vspace{-2pt}\item
    \begin{tabular*}{0.97\textwidth}[t]{l@{\extracolsep{\fill}}r}
      \makecell{#1} & #2 \\
      \small#3 & \small #4 \\
    \end{tabular*}\vspace{-7pt}
}

\newcommand{\CVPub}[2]{
  \vspace{-2pt}\item
    \begin{tabular*}{0.97\textwidth}[t]{p{0.85\linewidth}@{\extracolsep{\fill}}r}
      \small#1 & #2 \\
    \end{tabular*}\vspace{-7pt}
}

\newcommand{\CVSubSubheading}[2]{
    \item
    \begin{tabular*}{0.97\textwidth}{l@{\extracolsep{\fill}}r}
      \small{#1} & #2 \\
    \end{tabular*}\vspace{-7pt}
}

\newcommand{\CVSubItem}[1]{\CVItem{#1}\vspace{-4pt}}

\renewcommand\labelitemii{$\vcenter{\hbox{\tiny$\bullet$}}$}

\newcommand{\CVSubHeadingListStart}{\begin{itemize}[leftmargin=0.5cm, label={}]}
% \newcommand{\resumeSubHeadingListStart}{\begin{itemize}[leftmargin=0.15in, label={}]} % Uncomment for US
\newcommand{\CVSubHeadingListEnd}{\end{itemize}}
\newcommand{\CVItemListStart}{\begin{itemize}}
\newcommand{\CVItemListEnd}{\end{itemize}\vspace{-5pt}}

%------------------------------------------------------------------------------
% CV STARTS HERE  %
%------------------------------------------------------------------------------
\begin{document}

%-----HEADING------------------------------------------------------------------
\begin{comment}
In Europe it is common to include a picture of ones self in the CV. Select which heading appropriate for the document you are creating.
\end{comment}

%\begin{minipage}[c]{0.05\textwidth}
%\-\
%\end{minipage}
%\begin{minipage}[c]{0.2\textwidth}
%\begin{tikzpicture}
    %\clip (0,0) circle (1.75cm);
    %\node at (0,-.7) {\includegraphics[width = 9cm]{portrait}}; 
    % if necessary the picture may be moved by changing the at (coordinates)
    % width defines the 'zoom' of the picture
%\end{tikzpicture}
%\hfill\vline\hfill
%\end{minipage}
%\begin{minipage}[c]{0.4\textwidth}
    %\textbf{\Huge \scshape{Charles Rambo}} \\ \vspace{1pt} 
    % \scshape sets small capital letters, remove if desired
    %\small{+1 123-456-7890} \\
    %\href{mailto:you@provider.com}{\underline{you@provider.com}}\\
    % Be sure to use a professional *personal* email address
    %\href{https://www.linkedin.com/in/charles-rambo/}{\underline{linkedin.com/in/charles-rambo}} \\
    % you should adjust you linked in profile name to be professional and recognizable
    %\href{https://github.com/fizixmastr}{\underline{github.com/fizixmastr}}
%\end{minipage}

% Without picture
\begin{center}
    \textbf{\Huge \scshape Bokyung Lee} \\ \vspace{1pt} %\scshape sets small capital letters, remove if desired
    \small +1 647-570-5603 $|$ +82 10 3650 7644 \\
    \href{mailto:bokyung.lee@autodesk.com}{bokyung.lee@autodesk.com} $|$    \href{https://boleeHCI.com}{https://boleeHCI.com} \\
    % you should adjust you linked in profile name to be professional and recognizable
    \small Autodesk, 661 University Ave, Toronto, Ontario, Canada M5G 1M1
\end{center}



\begin{comment}
This CV was written for specifically for positions I was applying for in
academia, and then modified to be a template.

A standard CV is about two pages long where as a resume in the US is one page.
sections can be added and removed here with this in mind. In my experience, 
education, and applicable work experience and skills are the most import things
to include on a resume. For a CV the Europass CV suggests the categories: Work
Experience, Education and Training, Language Skills, Digital Skills,
Communication and Interpersonal Skills, Conferences and Seminars, Creative Works
Driver's License, Hobbies and Interests, Honors and Awards, Management and
Leadership Skills, Networks and Memberships, Organizational Skills, Projects,
Publications, Recommendations, Social and Political Activities, Volunteering.

Your goal is to convey a who, what , when, where, why for every item you share. 
The who is obviously you, but I believe the rest should be done in that order.
For example below. An employer cares most about the degree held and typically 
less about the institution or where it is located (This is still good 
information though). Whatever order you choose be consistent throughout.
\end{comment}

%-----EDUCATION----------------------------------------------------------------
\section{Education}
  \CVSubHeadingListStart
%    \CVSubheading % Example
%      {Degree Achieved}{Years of Study}
%      {Institution of Study}{Where it is located}
    \CVSubheading
      {{Korea Advanced Institute of Science and Technology (KAIST)}}{Mar. 2015 -- Feb 2020}
      {MA.\& Ph.D. of Engineering $|$ \emph{\small{Industrial Design}}}{Daejeon, South Korea}
      \CVItemListStart
        \CVItem{Embodied Design Toolkit for VR/AR}
        \CVItem{Interactive Human Bodies in Design Tools}
      \CVItemListEnd
    \CVSubheading
      {{TU Delft}}{Sep. 2014 -- Feb 2015}
      {M.A. Exchange Program $|$ \emph{\small{Department of Design Engineering}}}{Delft, Netherlands}
    \CVSubheading
      {{Aalto University}}{Jan. 2012 -- Aug 2012}
      {Undergraduate Exchange Program $|$ \emph{\small{School of Science}}}{Helsinki, Finland}
    \CVSubheading
      {Korea Advanced Institute of Science and Technology (KAIST)}{Feb. 2009 -- Feb 2014}
      {Bachelor of Engineering $|$ \emph{\small{Industrial Design \& Technology Management}}}{Daejeon, South Korea}
  \CVSubHeadingListEnd

%-----WORK EXPERIENCE----------------------------------------------------------
\begin{comment}
try to briefly explain what you did and why it is relevant to the position you are seeking
\end{comment}

\section{Work Experience}
  \CVSubHeadingListStart
%    \CVSubheading %Example
%      {What you did}{When you worked there}
%      {Who you worked for}{Where they are located}
%      \CVItemListStart
%        \CVItem{Why it is important to this employer}
%      \CVItemListEnd
    \CVSubheading
      {Autodesk Research $|$ \emph{\small{Simulation, Optimization, and Systems Group}}}{Mar 30. 2020 -- current}
      {Industrial Postdoctoral Fellow}{Toronto, Canada}
      \CVItemListStart
        \CVItem{Digital-Physical Convergence Systems/UX}
        \CVItem{UX for AI Virtual humans in the digital world}
        \CVItem{Human-centered AI-driven design tools}
        %\CVItem{Human-centered generative Design}
        
    \CVItemListEnd
    
    \CVSubheading
      {Autodesk Research $|$ \emph{\small{Complex Systems Group}}}{Nov 6. 2018 -- Aug 2. 2019}
      {Research Intern}{Toronto, Canada}
      \CVItemListStart
        \CVItem{Computer Vision-based Information Analysis Approach for Human-Building Interactions}
        \CVItem{Skeletonized ethnography tool}
      \CVItemListEnd
  \CVSubHeadingListEnd

%-----HONORS AND AWARDS--------------------------------------------------------
\section{Honors and Awards}
  \CVSubHeadingListStart
%    \CVSubheading %Example
%      {What}{When}
%      {Short Description}{}
    \CVSubheading
      {ACM Honorable Mention Award (Top 5\%)}{2021}
      {ACM Designing Interactive Systems Conference}{}
    \CVSubheading
      {ACM Best Paper Award (Top 1\%)}{2020}
      {ACM Hum.-Comput. Interact. Computer Supported Cooperative Work and Social Computing.}{}
    \CVSubheading
      {ACM Honorable Mention Award (Top 5\%)}{2019}
      {ACM Hum.-Comput. Interact. Computer Supported Cooperative Work and Social Computing.}{}
    \CVSubheading
      {Gender Equality Pictogram Bronze Award}{2014}
      {hosted by Choongchung City, Korea}{}
    \CVSubheading
      {Location-Based Service (LBS) Web \& Application Contest}{2013}
      {hosted by Korea Communication and Commissions $|$ organized by Korea Internet \& Security Agency}{}
    \CVSubheading
      {Innoplolis Idea Contest Grand Prize}{2013}
      {hosted by Ministry of Knowledge Economy of Korea}{}
  \CVSubHeadingListEnd
  
%-----HONORS AND AWARDS--------------------------------------------------------
\section{Patents}
  \CVSubHeadingListStart
%    \CVSubheading %Example
%      {What}{When}
%      {Short Description}{}
    \CVSubheading
      {Furniture Design Apparatus Displaying Personalized Guide Information}{Jan 2018}
      {Invention field: ICT/SW, KR-10-2018-0002957}{}
  \CVSubHeadingListEnd


%-----PROJECTS AND RESEARCH----------------------------------------------------
\begin{comment}
Ideally the title of the work should speak for what it is. However if you feel
like you should explain more about why the project is applicable to this job,
use item list as is shown in the work experience section.
\end{comment}

\section{Academic Achievements}    
  \CVSubHeadingListStart
%    \CVSubheading
%      {Title of Work}{When it was done}
%      {Institution you worked with}{unused}

    \CVPub
      {\underline{Bokyung Lee}, Michael Lee, Jacky Bibliowicz, Rhys Goldstein, Jeremy Mogk,  Alexander Tessier, \textbf{Simulation and Visualization of Virus Transmission for Architectural Design Analysis}, ACM \textbf{\textit{SIGGRAPH}}}{2021}
    
    \CVPub
      {\underline{Bokyung Lee}, Michael Lee, Jeremy Mogk, Rhys Goldstein, Jacky Bibliowicz, Frederik Brudy, Alexander Tessier, \textbf{Designing a Multi-Agent Occupant Simulation System to Support Facility Planning and Analysis for COVID-19}, ACM \textbf{\textit{DIS}} (Designing Interactive Systems)/\textcolor{blue}{Honorable Mention Award (Top 5\%)}}{}
    
    \CVPub
      {\underline{Bokyung Lee}, Michael Lee, Pan Zhang, Alexander Tessier, Daniel Saakes, Azam Khan., \textbf{Socio-Spatial Comfort: Using Vision-based Analysis to Inform User-Centred Human-Building Interactions}, ACM Hum.-Comput. Interact. 4, \textbf{\textit{CSCW3}}, Article 238 (December 2020), 33 pages./\textcolor{blue}{Best Paper Award (Top 1\%)}}{2020}
      
    \CVPub
      {Jaeho Sung, \underline{Bokyung Lee}, Daniel Saakes, \textbf{PoseScape: Pose-based Analysis System for Long-term Observation Studies}, ACM \textbf{\textit{NordiCHI 2020}} poster}{}
    
    \CVPub
      {\underline{Bokyung Lee}, Michael Lee, Pan Zhang, Alexander Tessier, Azam Khan., \textbf{An Empirical Study of How Socio-Spatial Formations are influenced by Interior Elements and Displays in an Office Context.} Proc. ACM Hum.-Comput. Interact. 3, \textbf{\textit{CSCW}}, Article 58./\textcolor{blue}{Honorable Mention Award (Top 5\%)}}{2019}
    
    \CVPub
      {\underline{Bokyung Lee}, Michael Lee, Alexander Tessier, Azam Khan., Skeletonographer: Skeleton-based Digital Ethnography Tool, ACM \textbf{\textit{CSCW 2019}} demo}{}
      
    \CVPub
      {Alexander Tessier, Simon Breslav, Kean Walmsley, Michael Lee, Hali Larsen, Jacky Bibliowicz, Pan Zhang, Liviu-Mihai Calin, \underline{Bokyung Lee}, Josh Cameron, Rhys Goldstein, and Azam Khan., \textbf{Occupancy Visualization towards Activity Recognition}, ACM \textbf{\textit{DFSH 2019}}}{}
      
    \CVPub
      {\underline{Bokyung Lee}, Michael Lee, Alexander Tessier, Azam Khan., \textbf{Semantic Human Activity Annotation Tool Using Skeletonized Surveillance Videos}, ACM \textbf{\textit{Ubicomp 2019}} demo}{}
      
    \CVPub
      {\underline{Bokyung Lee}, Taeil Jin, Sung-Hee Lee, Daniel Saakes., \textbf{SmartManikin: Virtual Humans with Agency for Design Tools}, ACM \textbf{\textit{CHI 2019}} full paper}{}
      
    \CVPub
      {\underline{Bokyung Lee}, Sindy Wu, Maria Reyes, Daniel Saakes., \textbf{The Effect of Interruption Timings on Autonomous Height-Adjustable Desks that Responds to Task Changes}, ACM \textbf{\textit{CHI 2019}} full paper}{}

    \CVPub
      {\underline{Bokyung Lee}, Joongi Shin, Hyoshin Bae, Daniel Saakes., \textbf{Interactive and Situated Guidelines to Help Users Design a Personal Desk that Fits Their Bodies}, ACM \textbf{\textit{DIS 2018}} full paper}{2018}
      
    \CVPub
      {\underline{Bokyung Lee}, Jiwoo Hong, Jaeheung Surh, Daniel Saakes., \textbf{Ori-mandu: Korean Dumpling into Whatever Shape You Want}, ACM \textbf{\textit{DIS 2017}} full paper (pictorial)}{2017}
      
    \CVPub
      {\underline{Bokyung Lee}, Gyeol Han, Jundong Park, Daniel Saakes., \textbf{Consumer to Creator: How Households Buy Furniture to Inform Design and Fabrication Interfaces.}, ACM \textbf{\textit{CHI 2017}} full paper}{}
      
    \CVPub
      {\underline{Bokyung Lee}, Jiwoo Hong, Jaeheung Surh, Daniel Saakes., \textbf{Ori-mandu: Korean Dumpling into Whatever Shape You Want}, ACM \textbf{\textit{CHI 2017}} video showcase}{}
      
    \CVPub
      {Jundong Park, \underline{Bokyung Lee}, Gyeol Han, Daniel Saakes., \textbf{Two Mental Models of Non-Professional Design Process for Future Fabrication Interface.}, \textbf{\textit{HCI Korea 2017}} full paper}{}
      
    \CVPub
      {\underline{Bokyung Lee}, Minjoo Cho, Daniel Saakes., \textbf{Posing and Acting as Input for Personalizing Furniture}, ACM \textbf{\textit{NordiCHI 2016}} full paper}{2016}
      
    \CVPub
      {SeungRyoel Kim, \underline{Bokyung Lee}, Daniel Saakes.,\textbf{Gesture-based Trafficator to Improve Driver to Traffic Communication.}, \textbf{\textit{KSDS 2016}} Fall 2016 paper}{}
      
    \CVPub
      {Foong-Yi Chia, \underline{Bokyung Lee}, Daniel Saakes., \textbf{Collaboration in the Living Room or How Couples Design Together}, ACM \textbf{\textit{NordiCHI 2016}} poster}{}
      
    \CVPub
      {Minjoo Cho, \underline{Bokyung Lee}, Joonhee Min, Daniel Saakes., \textbf{Sketching in Virtual Reality for Rapid and Situated Idea Generation}, \textbf{\textit{KSDS 2015}} Fall paper}{2015}
      
    \CVPub
      {\underline{Bokyung Lee}, Froukje Sleeswijk Visser, Daniel Saakes., \textbf{Online User Reviews as a Design Resource}, \textbf{\textit{IASDR}} 2015 full paper}{}
      
    \CVPub
      {Shin H-S (Felix), \underline{Bokyung Lee}, Daniel Saakes., \textbf{TagRadar: Locating Objects Using a Smart Phone Accessory}, ACM \textbf{\textit{Ubicomp}}/ISWC 2015 Demos}{}
  \CVSubHeadingListEnd


 
 
%-----CONFERENCES AND PRESENTATIONS--------------------------------------------
\begin{comment}
Again the title should have already been enough, but if it is necessary to add
descriptions maintain the consistency from prior sections
\end{comment}

\section{Invited Talks}
  \CVSubHeadingListStart
%    \CVSubheading % Example
%      {Work Presented}{When}
%      {Occasion}{}
    \CVSubheading
      {University of Toronto-Invited Talk: Dynamics Graphics Project Lab}{Mar 2021}
      {Title: Moving Forward our Embodied Interactions}{}
    \CVSubheading
      {HCI Korea Panel Talk: Panel for Future of Digital Fabrication}{Jan 2021}
      {Title: Not just helping you design easily, but help you design what you want.}{}
    \CVSubheading
      {ID KAIST Ph.D. Colloquium Talk}{Jul 2017}
      {Doing a Ph.D. in HCI \& Design Research}{}
    \CVSubheading
      {Autodesk Research Talk}{Nov 2018}
      {Embodied Design Tools for Digital Fabrication}{}
    \CVSubheading
      {KAIST CS374: Introduction to HCI Research Talk}{Jun. 2018}
      {Design-driven HCI Research}{}
    \CVSubheading
      {HCI KAIST Research Talk}{Apr 2017}
      {Consumer to Creator: Understanding how people buy furniture to inform fabrication tools}{}
  \CVSubHeadingListEnd


%-----TEACHING EXPERIENCE------------------------------------------------------
\begin{comment}
Section is here as it applied to my application for positions in academia. 
Remember to tailor the resume for to the position.
\end{comment}

\section{Teaching Experience}
  \CVSubHeadingListStart
%    \CVSubheading
%      {What}{When}
%      {School}{Where}
    \CVSubSubheading
      {Democratization of Design - Yeonsei University (special lecturer)}{Spring 2021}
    \CVSubSubheading
      {Scientific Research Paper Writing - Yeonsei University (special lecturer)}{Spring 2021}
    \CVSubSubheading
      {Digital Design Fabrication (teaching assistant)}{Fall 2017}
    \CVSubSubheading
      {ID KAIST Academic Counselor}{Spring 2017}
    \CVSubSubheading
      {Digital Design Fabrication (teaching assistant)}{Fall 2016}
    \CVSubSubheading
      {Integrated Design – Undergraduate graduation project}{Spring 2016}
    \CVSubSubheading
      {Digital Design Fabrication (teaching assistant)}{Fall 2015}
    \CVSubSubheading
      {Space Design (teaching assistant)}{Spring 2015}
    \CVSubSubheading
      {Introduction to Industrial Design (teaching assistant)}{Spring 2014}
  \CVSubHeadingListEnd

%-----PROJECT EXPERIENCE----------------------------------------------------
\section{Project Experience}
  \CVSubHeadingListStart
    \CVSubSubheading
      {Living Lab and Digital Twins}{2020-current}
    \CVSubSubheading
      {Designing and Simulating Digital Humans in the Virtual Space for Space/Product Design Evaluation}{2020-current}
    \CVSubSubheading
      {Vision-based Social Interaction Analysis for Human-Centered Architecture}{2020-current}
    \CVSubSubheading
      {Design Toolkits for Everybody with VR/AR Technology $|$ \textit{project manager}}{2015-2018}
    \CVSubSubheading
      {AI/AR Personal Health Recommendations for Mobile Computing at Home $|$ \textit{project manager}}{2018}
    \CVSubSubheading
      {AI/AR Personal Health Recommendations for Mobile Computing at Home $|$ \textit{project lead}}{2018}
    \CVSubSubheading
      {Gesture-based Trafficator to Enhance Human-Automobile Interaction $|$ \textit{teaching assistant}}{2017}
    \CVSubSubheading
      {Augmented Reality Interface for the Internet of Things $|$ \textit{participating researcher}}{2016-2018}
    \CVSubSubheading
      {An Interactive Tabletop to Support Surveillance Scenarios with UAVs $|$ \textit{participating researcher}}{2015}
    \CVSubSubheading
      {Design for the Next Generation UX-Oriented Mobile Software Platform $|$ \textit{researcher}}{2013}
  \CVSubHeadingListEnd


%-----LEADERSHIP ACTIVITIES----------------------------------------------------
\section{Leadership Activities}
  \CVSubHeadingListStart
    \CVSubSubheading
      {My Design Lab Student Lead}{Mar 2014 -- Dec 2017}
    \CVSubSubheading
      {Ph.D. student representative. Industrial Design department, KAIST}{Sep 2016 - Aug 2017}
    \CVSubSubheading
      {Student organizing lead, Design 3.0 Forum}{Aug 2017}
    \CVSubSubheading
      {Student organizing lead, ID KAIST 30th anniversary event}{Aug 2017}
    \CVSubSubheading
      {K2 KAIST-Kyushu University Workshop}{Feb 2017}
  \CVSubHeadingListEnd

%-----ACADEMIC SERVICE----------------------------------------------------
\section{Academic Service}
  \CVSubHeadingListStart
    \CVSubSubheading
      {Academic Review: ACM CHI, Architecture and Design Journal}{2021}
    \CVSubSubheading
      {Academic Review: ACM CHI, ACM Siggraph E-tech, ACM DIS, ACM IDC}{2020}
    \CVSubSubheading
      {Academic Review: ACM CHI, ACM Siggraph Asia}{2019}
    \CVSubSubheading
      {Academic Review: ACM CHI, ACM CHI Play, ACM CSCW, ACM DIS, ACM TEI, ACM UIST}{2018}
    \CVSubSubheading
      {Academic Review: ACM DIS, ACM IDC, ACM C\&C}{2017}
  \CVSubHeadingListEnd

%-----SKILLS-------------------------------------------------------------------
\begin{comment}
This section is compressed from the various skills sections that Euro CV
recommends.
\end{comment}

\section{Skills}
 \begin{itemize}[leftmargin=0.5cm, label={}]
    \small{\item{
     \textbf{Languages}{: Korean (Native), English (fluent)} \\
     \textbf{Programming}{: C\#, Python (NumPy, SciPy, Matplotlib, Pandas), Python, Javascript} \\
     \textbf{Tools}{: Unity, Adobe Illustrator, Adobe Photoshop, Adobe Premier, Adobe Indesign, Autodesk Fushion 360, Autodesk Revit, Autodesk Dynamo} \\
     \textbf{Document Creation}{: Microsoft Office Suite, LaTex} \\
     \textbf{Research Methods}{: design research methods (contextual inquiry, FGI, situated interview, gesture elicitation study), quantitative analysis (SPSS, R), mixed-methods} \\
    }}
 \end{itemize}
    
%------------------------------------------------------------------------------
\end{document}